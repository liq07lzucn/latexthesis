% Appendix Template

\chapter{Running \texttt{L-BFGS-B-NS}} % Main appendix title

\label{AppendixA} % Change X to a consecutive letter; for referencing this appendix elsewhere, use \ref{AppendixX}

\lhead{Appendix A. \emph{Running \texttt{L-BFGS-B-NS}}} % Change X to a consecutive letter; this is for the header on each page - perhaps a shortened title

\lstset{language=bash,
  basicstyle=\ttfamily\scriptsize,
  keywordstyle=\color{blue},
  commentstyle=\color{magenta},
  morecomment=[l]{!\ }% Comment only with space after !
  backgroundcolor=\color{white},
  numbers=left,
  numbersep=5pt,                   % how far the line-numbers are from the code
  breaklines=true,
  firstnumber = 2607
}

\section{Running tests in local machines}

Running the software requires getting a copy of the files. The easiest way to get a copy is downloading directly from the repository \citep{lbfgsbNS} and clicking on the \textup{download zip} link. Users of git can also clone the repository by issuing either the Hypertext Transfer Protocol Secure command \textup{git clone https://github.com/wilmerhenao/L-BFGS-B-NS.git} or the secure shell command \textup{git clone git@github.com:wilmerhenao/L-BFGS-B-NS.git}. Other requirements on the machine are a \texttt{FORTRAN} compiler and a \texttt{LAPACK} package.

Once the user has obtained a local copy of the code. New executables need to be created. in order to do this, a simple \texttt{Makefile} has been provided, in order to ``make'' the executables and run a typical test with parameters $p = 1.1$, $n = 100$, $m = 5$ and $\tau_d = 10^{-6}$, the user should issue the following commands.

\begin{lstlisting}[language=bash]
  $ make
  $ ./rosenbrockp 1.1 100 5 1d-6
\end{lstlisting}

Output by default goes directly to the screen. The best way to capture the results in a text file is using the redirect operator >.

\begin{lstlisting}[language=bash]
  $ make
  $ ./rosenbrockp 1.1 100 5 1d-6 > mysampleresults.txt
\end{lstlisting}

If the user is running several tests, a bash script might be necessary.

\begin{lstlisting}[language=bash]{runbatch.sh}
\#\!/bin/bash
for ptol in 1d-6
do
    for p in 2 1 1.5 1.1 1.01 1.001 1.0001 1.00001 0.99 0.9
    do
	for n in 2 4 6 8 10 20 50 100 200 1000 5000 10000
	do
	    for m in 5 10 20
	    do
		echo \$ptol \$p \$n \$m
		./rosenbrockp \$p \$n \$m \$ptol >> OUTPUTS/res1d6.txt
	    done
	done
    done
done

exit 0;
\end{lstlisting}

This bash script can be made executable and run directly on the user's machine. All the results from the runs will be located on file \texttt{OUTPUTS/resid6.txt}

\begin{lstlisting}[language=bash]
  $ chmod +x runall.sh
  $ ./runbatch.sh
\end{lstlisting}

\section{Running on High Performance Computer Clusters}

The requirements are different on the high performance computing cluster, but the standard is to use Portable Batch System \emph{PBS} files. They allow the user to get detailed information of the tests via e-mails and provide the user with the ressources to run larger problems.

\begin{lstlisting}[language=bash]{precision1d6.pbs}
  #!/bin/bash

  #PBS -l nodes=1:ppn=8,walltime=48:00:00
  #PBS -m abe
  #PBS -M youremail@nyu.edu
  #PBS -N rosenbrockHD9

  module load gcc/4.7.3

  cd /scratch/weh227/rosenbrock/
  for ptol in 1d-6
  do
    for p in 2 1 1.5 1.1 1.01 1.001 1.0001 1.00001 0.99 0.9
    do
	for n in 2 4 6 8 10 20 50 100 200 1000 5000 10000 100000 1000000
	do
	    for m in 5 10 20
	    do
		echo \$ptol \$p \$n \$m
		./rosenbrockp \$p \$n \$m \$ptol >> res1d6.txt
	    done
	done
    done
  done

  exit 0;
\end{lstlisting}

Typically the user logs into the clusters, runs the tests usign a software called \emph{qsub} specially for this task and picks up the results once the tests have finished. At NYU user log in to the hpc access cluster, followed by the bowery computer cluster.

\begin{lstlisting}
  $ ssh youremail@hpc.nyu.edu
  $ ssh bowery
  $ password: ********
  $ git clone https://www.github.com/wilmerhenao/L-BFGS-B-NS.git
  $ qsub -o precision1d6.log -j oe precision1d6.pbs
\end{lstlisting}

And of course it is possible to run several \emph{PBS} jobs at once. Your local High Performance Computer Cluster always has some documentation on how to create and use \emph{PBS} files. https://wikis.nyu.edu/display/NYUHPC/Tutorial+-+Submitting+a+job+using+qsub
