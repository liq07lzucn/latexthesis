% Appendix Template

\chapter{Running \texttt{L-BFGS-B-NS}} % Main appendix title

\label{AppendixA} % Change X to a consecutive letter; for referencing this appendix elsewhere, use \ref{AppendixX}

\lhead{Appendix A. \emph{Running \texttt{L-BFGS-B-NS}}} % Change X to a consecutive letter; this is for the header on each page - perhaps a shortened title

\lstset{language=bash,
  basicstyle=\ttfamily\scriptsize,
  keywordstyle=\color{blue},
  commentstyle=\color{magenta},
  morecomment=[l]{!\ }% Comment only with space after !
  backgroundcolor=\color{white},
  %numbers=left,
  %numbersep=5pt,                   % how far the line-numbers are from the code
  breaklines=true,
  %firstnumber = 2607
}

\section{Running tests in local machines}

In order to run the software, a copy of the files is required. The easiest way to get a copy is downloading directly from the repository \citep{lbfgsbNS} and clicking on the \textup{download zip} link. Users of git can also clone the repository by issuing either the Hypertext Transfer Protocol Secure command \texttt{git clone https://github.com/wilmerhenao/L-BFGS-B-NS.git} or the secure shell command \texttt{git clone git@github.com:wilmerhenao/L-BFGS-B-NS.git}. Other requirements on the machine are a \texttt{FORTRAN} compiler and \texttt{LAPACK}.

Once the user has obtained a local copy of the code, new executables need to be created. A simple \texttt{Makefile} has been provided; in order to ``make'' the executables and run a typical test with parameters $p = 1.1$, $n = 100$, $m = 5$ and $\tau_d = 10^{-6}$, the user should issue the following commands.

\begin{lstlisting}[language=bash]
  $ make
  $ ./rosenbrockp 1.1 100 5 1d-6
\end{lstlisting}

Output by default goes directly to the screen. The best way to capture the results in a text file is using the redirection operator $>$.

\begin{lstlisting}[language=bash]
  $ make
  $ ./rosenbrockp 1.1 100 5 1d-6 > mysampleresults.txt
\end{lstlisting}

If the user is running several tests, a bash script might be necessary.

\begin{lstlisting}[language=bash]{runbatch.sh}
#!/bin/bash
for ptol in 1d-6
do
    for p in 2 1 1.5 1.1 1.01 1.001 1.0001 1.00001 0.99 0.9
    do
	for n in 2 4 6 8 10 20 50 100 200 1000 5000 10000
	do
	    for m in 5 10 20
	    do
		echo \$ptol \$p \$n \$m
		./rosenbrockp \$p \$n \$m \$ptol >> OUTPUTS/res1d6.txt
	    done
	done
    done
done

exit 0;
\end{lstlisting}

This bash script can be made executable and run directly on the user's machine. All the results from the runs will be located on file \texttt{OUTPUTS/resid6.txt} in this case.

\begin{lstlisting}[language=bash]
  $ chmod +x runall.sh
  $ ./runbatch.sh
\end{lstlisting}

\section{Running on High Performance Computer Clusters}

The requirements are different on the high performance computing cluster, but the standard is to use Portable Batch System \emph{PBS} files. They allow the user to get detailed information of the tests via e-mails and provide the user with the ressources to run larger problems.

\begin{lstlisting}[language=bash]{precision1d6.pbs}
  #!/bin/bash

  #PBS -l nodes=1:ppn=8,walltime=48:00:00
  #PBS -m abe
  #PBS -M youremail@nyu.edu
  #PBS -N rosenbrockHD9

  module load gcc/4.7.3

  cd /scratch/weh227/rosenbrock/
  for ptol in 1d-6
  do
    for p in 2 1 1.5 1.1 1.01 1.001 1.0001 1.00001 0.99 0.9
    do
	for n in 2 4 6 8 10 20 50 100 200 1000 5000 10000 100000 1000000
	do
	    for m in 5 10 20
	    do
		echo $ptol $p $n $m
		./rosenbrockp $p $n $m $ptol >> res1d6.txt
	    done
	done
    done
  done

  exit 0;
\end{lstlisting}

Typically the user logs into the clusters, runs the tests using a software called \emph{qsub} and picks up the results once the tests have finished. At NYU a user logs into the hpc access cluster, followed by the bowery computer cluster.

\begin{lstlisting}
  $ ssh youremail@hpc.nyu.edu
  $ ssh bowery
  $ password: ********
  $ git clone https://www.github.com/wilmerhenao/L-BFGS-B-NS.git
  $ cd L-BFGS-B-NS
  $ make
  $ qsub -o precision1d6.log -j oe precision1d6.pbs
\end{lstlisting}

It is also possible to run several \emph{PBS} jobs simultaneously. Your local High Performance Computer Cluster always has some documentation on how to create and use \emph{PBS} files. https://wikis.nyu.edu/display/NYUHPC/Tutorial+-+Submitting+a+job+using+qsub

\section{Specifying the function and the Gradient}

The user can specify a new function and a its corresponding gradient. Just change lines $219$ to $232$ in file \texttt{DriverRosenbrockp.f90} 

\lstset{language=[90]Fortran,
  basicstyle=\ttfamily\scriptsize,
  keywordstyle=\color{blue},
  commentstyle=\color{magenta},
  morecomment=[l]{!\ }% Comment only with space after !
  backgroundcolor=\color{white},
  numbers=left,
  numbersep=5pt,                   % how far the line-numbers are from the code
  breaklines=true,
  firstnumber = 219
}

\begin{lstlisting}
  f=((x(1)-1d0)**2)
  g(1)=2d0*(x(1)-1d0)
  
  do 20 i=1, (n-1)
    z=x(i+1)-x(i)**2
    f=f+abs(z)**p
    !r1=p*(z**2)**(p/2)/z
    r1=p * abs(z)**(p - 1)
    if (z < 0) then
      r1 = -r1
    endif
    g(i+1)=r1
    g(i)=g(i)-2d0*x(i)*r1
20 continue
\end{lstlisting}

Boundaries and starting points are also specified in this file in between lines $95$ and $119$
