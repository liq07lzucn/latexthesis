\chapter{Experimental Results}
\lhead{Chapter 4. \emph{Experimental Results}}

The \texttt{L-BFGS-B} implementation was tested on the high performance cluster machines at NYU. In order to run these tests it was necessary to create a series of PBS files\footnote{PBS stands for Portable Batch System. This is software that performs job scheduling. It is used by High Performance Computing at NYU (and many other High Performance Computing Centers) to allocate computational tasks. In order to run jobs at the high performance clusters, a series of PBS batch files need to be created.} using a PBS generator script. This script generator created PBS files which in turn run bash shell scripts\footnote{Bash is a command processor. Each Bash script that was created includes a series of computer commands, namely, execution of the original \texttt{L-BFGS-B} software and the new code \texttt{L-BFGS-B-NS}.}. Several of these shell scripts are available at the repository \citep{lbfgsbNS}. The main reason to run scripts this way is because it achieves parallelism, and because the system sends confirmation e-mails and statistics about the different stages of the processes giving a lot of control to the practitioner.

\section{Exit Messages}

The original \texttt{L-BFGS-B} optimizer displays different messages depending on the condition that triggered the exit. The following is a list of some of the most common exit messages in the original \texttt{L-BFGS-B} optimizer along with an explanation of how we generalized them in \texttt{L-BFGS-B-NS}.

\begin{itemize}

\item ``ABNORMAL\_TERMINATION\_IN\_LNSRCH'' This message means that there was a problem with the line search and the program's exit was premature. In \texttt{L-BFGS-B}, it typically occurs for non-smooth functions where the line search breaks down. In \texttt{L-BFGS-B-NS}, it typically occurs when the limit on the number of bisections in the line search ($30$) is exceeded.

\item ``CONVERGENCE: NORM\_OF\_PROJECTED\_GRADIENT\_LT\_PGTOL": Means that convergence was achieved because the norm of the projected gradient is small enough. We made just one change to the original \texttt{L-BFGS-B} code: in order to have results that are comparable with the results obtained with the new code, we terminated when the $2$-norm of the projected gradient was less than the tolerance $\tau =10^{-6}$, instead of the $\infty$-norm. Notice that this convergence message does not apply to \texttt{L-BFGS-B-NS} because of particular requirements for non-smooth functions involving the convex hull of projected gradients as explained in section \ref{terminator}. Instead it is replaced by:

\item ``CONVERGENCE: ZERO\_GRAD\_IN\_CONV\_HULL" This means that the termination condition discussed in section \ref{terminator} was satisfied\footnote{This does not mean that the resulting vector is exactly equal to zero, but it is small enough to satisfy the termination condition.}. We set $\tau_d = 10^{-6}$ and $\tau_x = 10^{-3}$.

\item ``CONVERGENCE: REL\_REDUCTION\_OF\_F\_LT\_FACTR*EPSMCH": This convergence condition is achieved whenever the relative reduction of the value of function $f$ is smaller than a predefined factor times the machine precision $\epsilon$. This exit message does not apply to our tests. It was disabled by setting the factor ``FACTR'' to zero, both in our runs of \texttt{L-BFGS-B-NS} and our tests using the original code \texttt{L-BFGS-B}.

\end{itemize}

The limit on the number of iterations was set to $10000$.

\section{Modified Rosenbrock Function} \label{ros}

Consider a modified version of the Rosenbrock function problem \citep{rosenbrock}:

\begin{equation} \label{modifiedrosenbrock}
    f(x) = (x_1 - 1)^2 + \sum_{i = 2}^n |x_i - x_{i - 1}^2|^p
\end{equation}

We can study the properties of function $f$ based on the properties of the function $\phi(t_i)$, where $\phi(t_i) = |t_i|^p$ and $t_i = x_i - x_{i - 1}^2$. The properties of the function depend on the value of the $p$ parameter\footnote{The original Rosenbrock function had a value of $p = 2$ and the second term was multiplied by $100$.}. This function can be proven to be locally Lipschitz continuous whenever $p \geq 1$. However, its second derivative blows up at zero whenever $p < 2$. Note that although $\phi(t_i)$ is convex for $p \geq 1$, $f$ is not convex.

The properties of $\phi(t_i)$ can be separated into different cases. Whenever $p > 1$ the derivative can be represented as:
\begin{equation}\label{firstderiv}
  \frac{d}{dt} \phi(t) = \pm p |t|^{p-1}
\end{equation}
and therefore, the limit of the derivative exists and is equal to zero near $t = 0$: \[ \lim_{t \to 0} \frac{d}{dt}\phi(t) = 0 \] From this we conclude that $f$ has a smooth first derivative for $p > 1$.

However, if $p = 1$, $\phi(t) = |t|$, and the absolute value function is not differentiable at $t = 0$. Note that in this case, $\phi(t)$ is Lipschitz continuous at $t = 0$.

The second derivative provides a bit more of information.

\begin{equation}\label{secondderiv}
  \frac{d^2}{dt^2} \phi(t) = \pm p(p-1) |t|^{p-2}
\end{equation}

If $p \geq 2$ the function is twice continuously differentiable. However if $p < 2$, the second derivative becomes $\frac{p(p-1)}{|t|^{q}}$, where $q = 2 - p > 0$, and this second derivative blows up as $|t| \to 0$. The special case $p = 1$ has second derivative equal to zero since $p(p-1) = 0$ except at $t = 0$ where it is undefined. For $p < 1$, $\phi$ is not Lipschitz continuous at $t = 0$.

Having explained the characteristics of the function, the next thing that needs to be defined is the region to be tested. We chose the region to be defined by the ``box'' with boundaries

\begin{equation}
  \begin{aligned}
    x_i = 
    \begin{cases}
      [-100, 100] & \text{if } i \in \text{ even numbers} \\
      [10, 100] & \text{if } i \in \text{ odd numbers}
    \end{cases}
  \end{aligned}
\end{equation}

The initial point was chosen to be the midpoint of the box, plus a different small perturbation for each dimension, chosen so that the line search does not reach the boundary of several dimensions in one step:

\begin{equation}
  \begin{aligned}
    x_i = \frac{u_i + l_i}{2} - \left(1 - 2^{1 - i}\right)
  \end{aligned}
\end{equation}

It is important to note that this choice of initial point makes the problem more difficult to solve. The problem is easier if the midpoint is chosen.

We tested both the original \texttt{L-BFGS-B} optimizer and the new code \texttt{L-BFGS-B-NS} on the modified Rosenbrock function with $p$ varying between $2$ and $0.9$ and $n$ varying between $100$ and $10000$, with the variable memory parameter $m$ set to $5$, $10$ and $20$.

\begin{center}
  \begin{table}
    \begin{center}
      \scriptsize
      \begin{tabular}{|l|l|l|l|l|l|l|l|l|l|l|}
        \hline
      p &  n  &  m  & \multicolumn{4}{|c|}{\texttt{L-BFGS-B} results} & \multicolumn{4}{|c|}{\texttt{L-BFGS-B-NS} results} \\ \hline
        & &  & iters. & \#fg & f & NPG & iters. & \#fg & f & NSVCHPG \\ \hline
      2 &  100 & 5 & 29 & 131 & 452116.014385974 & 2.92E-06 & 34 & 67 & 452116.014385974 & 1.46E-08\\
      2 &  100 & 10  & 25 & 67 & 452116.014385974 & 7.37E-05 & 32 & 45 & 452116.014385974 & 2.29E-07\\
      2 &  100 & 20  & 25 & 28 & 452116.014385974 & 8.97E-07 & 29 & 47 & 452116.014385974 & 1.20E-04\\
      2 &  200 & 5 & 32 & 74 & 913376.515331672 & 1.97E-07 & 34 & 62 & 913376.515331677 & 8.44E-07\\
      2 &  200 & 10  & 27 & 32 & 913376.515331672 & 5.26E-07 & 32 & 43 & 913376.515331672 & 1.04E-07\\
      2 &  200 & 20  & 26 & 29 & 913376.515331672 & 5.80E-07 & 33 & 58 & 913376.515331677 & 3.98E-08\\
      2 &  1000 & 5  & 26 & 68 & 4603460.52289722 & 2.61E-04 & 37 & 80 & 4603460.52289732 & 9.85E-07\\
      2 &  1000 & 10  & 26 & 71 & 4603460.52289722 & 5.93E-04 & 33 & 45 & 4603460.52289733 & 5.89E-07\\
      2 &  1000 & 20  & 30 & 95 & 4603460.52289722 & 1.18E-05 & 33 & 59 & 4603460.52289732 & 9.02E-07\\
      2 & 5000 & 5 & 27 & 122 & 23053880.5607232 & 4.44E-03 & 23 & 41 & 23053880.5607253 & 2.19E-07\\
      2 & 5000 & 10 & 25 & 28 & 23053880.5607256 & 8.96E-07 & 32 & 40 & 23053880.5607253 & 8.40-07\\
      2 & 5000 & 20 & 26 & 80 & 23053880.560724 & 3.75E-03 & 17 & 43 & 23053880.5607232 & 1.08E-07\\
      2 & 10000 & 5 & 26 & 68 & 46116905.6080045 & 2.80E-06 & 12 & 232 & 46116905.6079994 & 5.41E-06\\
      2 & 10000 & 10 & 25 & 67 & 46116905.6080057 & 7.70E-03 & 18 & 297 & 46116905.6080044 & 3.66E-05\\
      2 & 10000 & 20 & 28 & 73 & 46116905.608006 & 5.18E-06 & 18 & 297 & 46116905.6080044 & 3.66E-05\\
      \hline
      \end{tabular}
      \caption[Modified Rosenbrock with $p = 2$]{Satisfactory results for the original algorithm \texttt{L-BFGS-B}  and for \texttt{L-BFGS-B-NS} applied to the Modified Rosenbrock function with $p = 2$. NPG: Norm of Projected Gradient with tolerance $10^{-6}$. 
NSVCHPG: Norm of Smallest Vector in Convex Hull of Projected Gradients with $\tau_d = 10^{-6}, \tau_x = 10^{-3}$}
      \label{pequal2}
    \end{center}
  \end{table}
\end{center}

\subsection{Performance of \texttt{L-BFGS-B} and \texttt{L-BFGS-B} on the Modified Rosenbrock Function}

In the tables, iters shows the number of iterations, $\#fg$ shows the number of function and gradient evaluations taken, and $f$ shows the final computed function value that was achieved by the optimization. In the case of \texttt{L-BFGS-B}, \emph{NPG} shows the $2$-norm of the projected gradient with termination tolerance $10^{-6}$. In the case of \texttt{L-BFGS-B-NS}, \emph{NSVCHPG} shows the norm of the smallest vector in the convex hull of projected gradients. 

In table \ref{pequal2} we can see that since the test function is smooth when $p = 2$, \texttt{L-BFGS-B} has no problems minimizing it and that \texttt{L-BFGS-B-NS} reaches exactly the same final values, although because of the line search changes, the number of iterations and function and gradient evaluations usually increase.

\begin{table}
  \scriptsize
  \begin{center}
    \begin{tabular}{|l|l|l|l|l|l|l|l|l|l|l|}
      \hline
      p  &  n  &  m  & \multicolumn{4}{|c|}{\texttt{L-BFGS-B} results} & \multicolumn{4}{|c|}{\texttt{L-BFGS-B-NS} results} \\ \hline
      & &  & iters. & \#fg & f & NPG & iters. & \#fg & f & NSVCHPG \\ \hline
      1 & 100 & 5  & 1 & 21 & 151292.8 & 7.79E+02 & 15 & 130 & 4826.1066601788 & 1.34E-08\\
      1 & 100 & 10  & 1 & 21 & 151292.8 & 7.79E+02 & 15 & 129 & 4826.1066352341 & 1.34E-08\\
      1 & 100 & 20  & 1 & 21 & 151292.8 & 7.79E+02 & 15 & 129 & 4826.1066352341 & 1.34E-08\\
      1 & 200 & 5  & 1 & 21 & 299792.8 & 1.10E+03 & 15 & 128 & 9668.0522943829 & 1.82E-08\\
      1 & 200 & 10  & 1 & 21 & 299792.8 & 1.10E+03 & 15 & 128 & 9668.0522930362 & 1.82E-08\\
      1 & 200 & 20  & 1 & 21 & 299792.8 & 1.10E+03 & 15 & 112 & 9667.9345180734 & 1.19E-07\\
      1 & 1000 & 5  & 1 & 21 & 1487792.8 & 2.46E+03 & 23 & 193 & 48403.1390323475 & 5.72E-09\\
      1 & 1000 & 10  & 1 & 21 & 1487792.8 & 2.46E+03 & 16 & 160 & 48403.3203939957 & 2.44E-08\\
      1 & 1000 & 20  & 1 & 21 & 1487792.8 & 2.46E+03 & 16 & 160 & 48403.320394002 & 2.44E-08\\
      1 & 5000 & 5 & 1 & 21 & 7427792.8 & 5.51E+03 & 20 & 127 & 242078.712084738 & 1.26E-08\\
      1 & 5000 & 10 & 1 & 21 & 7427792.8 & 5.51E+03 & 56 & 339 & 242078.839910433 & 1.26E-08\\
      1 & 5000 & 20 & 1 & 21 & 7427792.8 & 5.51E+03 & 45 & 249 & 242078.560631846 & 7.84E-08\\
      1 & 10000 & 5 & 1 & 21 & 14852792.8 & 7.79E+03 & 18 & 148 & 484172.781463252 & 8.25E-08\\
      1 & 10000 & 10 & 1 & 21 & 14852792.8 & 7.79E+03 & 10000 & 20019 & 484269.73074638832 & 3.76E+02 \\
      1 & 10000 & 20 & 1 & 21 & 14852792.8 & 7.79E+03 & 21 & 101 & 484172.918293261 & 1.77E-08\\
      \hline
    \end{tabular}
    \caption[Modified Rosenbrock with $p = 1$]{Unsatisfactory results for the original algorithm \texttt{L-BFGS-B} applied to the Modified Rosenbrock function with $p = 1$, but converging results for \texttt{L-BFGS-B-NS}. NPG: Norm of Projected Gradient with tolerance = $10^{-6}$. NSVCHPG: Norm of Smallest Vector in Convex Hull of Projected Gradients with $\tau_d = 10^{-6}, \tau_x = 10^{-3}$}
    \label{pequal1merged}
  \end{center}
\end{table}

On the other hand, in table \ref{pequal1merged} we see that the value of $p = 1$ leads to an abnormal line search termination for \texttt{L-BFGS-B} in all of the cases presented. This is to be expected as the function is non-smooth. In fact, for all cases \texttt{L-BFGS-B} terminates at the first iteration because of breakdown in the line search.  \texttt{L-BFGS-B-NS} on the other hand is able to converge under most scenarios. In fact, it is possible to converge under all scenarios by tweaking the starting point of the optimization.

Several other values of $p$ were also tested. In Table \ref{pmtable}, the parameter $p$ is varied and all other parameters are held constant. With a tolerance of $10^{-6}$ \texttt{L-BFGS-B} always fails as expected, and those values where $p$ is closer to $1$ are the most difficult for the original algorithm to handle.  Values generated via \texttt{L-BFGS-B-NS} are comparatively better whenever $p < 2$, since the function is ``less'' smooth. Most runs of \texttt{L-BFGS-B-NS} converge using the termination condition from section \ref{terminator}.

\begin{table}
  \tiny
  \begin{center}
    \begin{tabular}{|l|l|l|l|l|l|l|l|l|l|l|}
      \hline
      p & n & m  & \multicolumn{4}{|c|}{\texttt{L-BFGS-B} results} & \multicolumn{4}{|c|}{\texttt{L-BFGS-B-NS} results} \\ \hline
      &  & & iters. & \#fg & f & NPG & iters. & \#fg & f & NSVCHPG \\ \hline
      2 & 200 & 5 & 32 & 74 & 913376.515331672 & 1.97E-07 & 35 & 55 & 913376.515331676 & 3.19E-09\\
      2 & 200 & 10  & 27 & 32 & 913376.515331672 & 5.26E-07 & 20 & 41 & 913376.515331677 & 3.98E-07\\
      2 & 200 & 20 & 26 & 29 & 913376.515331672 & 5.80E-07 & 19 & 40 & 913376.515331672 & 8.03E-07\\
      1.5 & 200 & 5 &  8 & 50 & 95144.1877450699 & 9.60E+02 & 29 & 68 & 94261.6310280216 & 7.52E-07\\
      1.5 & 200 & 10 &  8 & 50 & 95095.5635531693 & 9.61E+02 & 30 & 59 & 94261.6310280212 & 9.69E-07\\
      1.5 & 200 & 20 &   8 & 50 & 95095.5635531693 & 9.61E+02 & 26 & 66 & 94261.6310280211 & 9.95E-07\\
      1.1 & 200 & 5 &  1 & 21 & 658485.96769483 & 1.10E+03 & 26 & 75 & 15226.525226329 & 4.24E-07\\
      1.1 & 200 & 10 &  1 & 21 & 658485.96769483 & 1.10E+03 & 34 & 107 & 15226.5210644821 & 1.16E-07\\
      1.1 & 200 & 20 &  1 & 21 & 658485.96769483 & 1.10E+03 & 38 & 99 & 15226.5209960549 & 1.73E-07\\
      1.01 & 200 & 5 &  1 & 21 & 324235.017102379 & 1.10E+03 & 31 & 305 & 10218.0196721806 & 3.64E+01\\
      1.01 & 200 & 10 &  1 & 21 & 324235.017102379 & 1.10E+03 & 47 & 151 & 10116.5275434197 & 7.29E-07\\
      1.01 & 200 & 20 &  1 & 21 & 324235.017102379 & 1.10E+03 & 29 & 123 & 10116.5603888173 & 2.95E-09\\
      1.001 & 200 & 5 &  1 & 21 & 302150.58179968 & 1.10E+03 & 36 & 111 & 9711.8763115237 & 5.70E-08\\
      1.001 & 200 & 10 &  1 & 21 & 302150.58179968 & 1.10E+03 & 23 & 100 & 9711.8906439951 & 2.81E-09\\
      1.001 & 200 & 20 &  1 & 21 & 302150.58179968 & 1.10E+03 & 39 & 164 & 9711.876311317 & 1.41E-07\\
      1.0001 & 200 & 5 &  1 & 21 & 300027.736327598 & 1.10E+03 & 306 & 638 & 9672.3210642275 & 5.09E-07\\
      1.0001 & 200 & 10 &  1 & 21 & 300027.736327598 & 1.10E+03 & 17 & 96 & 9672.3639815678 & 1.82E-08\\
      1.0001 & 200 & 20 &  1 & 21 & 300027.736327598 & 1.10E+03 & 19 & 96 & 9672.3922445339 & 2.80E-09\\
      1.00001 & 200 & 5 &  1 & 21 & 299816.285236336 & 1.10E+03 & 27 & 96 & 9668.3934739514 & 4.32E-07\\
      1.00001 & 200 & 10 &  1 & 21 & 299816.285236336 & 1.10E+03 & 15 & 80 & 9668.373073478 & 2.80E-09\\
      1.00001 & 200 & 20 &  1 & 21 & 299816.285236336 & 1.10E+03 & 15 & 80 & 9668.3730743134 & 2.80E-09\\
      1 & 200 & 5 & 1 &  21  & 299792.8 & 1.10E+03 & 15 & 128 & 9668.0522943829 & 1.82E-08\\
      1 & 200 & 10 & 1 &  21 & 299792.8 & 1.10E+03 & 15 & 128 & 9668.0522930362 & 1.82E-08\\
      1 & 200 & 20 & 1 &  21 & 299792.8 & 1.10E+03 & 15 & 112 & 9667.9345180734 & 1.19E-07\\
      \hline
    \end{tabular}
    \caption[Number of algorithm Iterations Changing $p$]{Results for different values of $p$, with $n = 200$.}
    \label{pmtable}
  \end{center}
\end{table}

Some runs with a value of $p = 0.999$ are shown in Table \ref{p0999}. As can be seen, both algorithms fail in the sense that the termination criteria are never met, but \texttt{L-BFGS-B-NS} reaches a better feasible solution in every scenario. The same thing can be said for $p = 0.99$ on table \ref{p099} and for $p = 0.9$ on table \ref{p09}.

\begin{table}
  \tiny
  \begin{center}
    \begin{tabular}{|l|l|l|l|l|l|l|l|l|l|l|}
      \hline
      p  & n & m  & \multicolumn{4}{|c|}{\texttt{L-BFGS-B} results} & \multicolumn{4}{|c|}{\texttt{L-BFGS-B-NS} results} \\ \hline
      &  &  & iters. & \#fg & f & NPG & iters. & \#fg & f & NSVCHPG \\ \hline
      0.999 & 100 & 5 & 1 & 21 & 150123.179035242 & 7.79E+02 & 10000 & 20003 & 4900.9128213197 & 3.86E+01\\
      0.999 & 100 & 10 & 1 & 21 & 150123.179035242 & 7.79E+02 & 10000 & 19999 & 4900.9123782223 & 3.79E+01\\
      0.999 & 100 & 20 & 1 & 21 & 150123.179035242 & 7.79E+02 & 10000 & 20000 & 4900.8873111184 & 3.78E+01\\
      0.999 & 200 & 5 & 1 & 21 & 297453.671572579 & 1.10E+03 & 10000 & 29971 & 9720.7074076621 & 5.50E+01\\
      0.999 & 200 & 10 & 1 & 21 & 297453.671572579 & 1.10E+03 & 10000 & 19999 & 9720.7073593488 & 5.41E+01\\
      0.999 & 200 & 20 & 1 & 21 & 297453.671572579 & 1.10E+03 & 10000 & 20000 & 9720.7067337013 & 5.39E+01\\
      0.999 & 1000 & 5 & 1 & 21 & 1476097.61187127 & 2.46E+03 & 10000 & 29961 & 48279.0637949643 & 9.94E-01\\
      0.999 & 1000 & 10 & 1 & 21 & 1476097.61187127 & 2.46E+03 & 10000 & 20000 & 48279.0637881564 & 1.68E+02\\
      0.999 & 1000 & 20 & 1 & 21 & 1476097.61187127 & 2.46E+03 & 10000 & 20000 & 48279.0637186514 & 1.66E+02\\
      0.999 & 5000 & 5 & 1 & 21 & 7369317.31336543 & 5.51E+03 & 10000 & 29983 & 241070.845631957 & 9.94E-01\\
      0.999 & 5000 & 10 & 1 & 21 & 7369317.31336543 & 5.51E+03 & 10000 & 29983 & 241070.845630635 & 9.94E-01\\
      0.999 & 5000 & 20 & 1 & 21 & 7369317.31336543 & 5.51E+03 & 10000 & 20005 & 241070.845626631 & 2.73E+02\\
      0.999 & 10000 & 5 & 1 & 21 & 14735841.9402302 & 7.79E+03 & 10000 & 29981 & 482060.572922137 & 3.89E+02\\
      0.999 & 10000 & 10 & 1 & 21 & 14735841.9402302 & 7.79E+03 & 10000 & 29983 & 482060.572921515 & 9.94E-01\\
      0.999 & 10000 & 20 & 1 & 21 & 14735841.9402302 & 7.79E+03 & 10000 & 20003 & 482060.572910768 & 5.28E+02\\
      \hline
    \end{tabular}
    \caption[A value where \texttt{L-BFGS-B-NS} is supposed to fail. $p = 0.999$]{Results for $p = 0.999$, and non-converging but better results for \texttt{L-BFGS-B-NS}; NPG: Norm of Projected Gradient with tolerance = $10^{-6}$, never satisfied. NSVCHPG: Norm of Smallest Vector in Convex Hull of Projected Gradients with $\tau_d = 10^{-6}, \tau_x = 10^{-3}$, also, never satisfied}
    \label{p0999}
  \end{center}
\end{table}

\begin{table}
  \tiny
  \begin{center}
    \begin{tabular}{|l|l|l|l|l|l|l|l|l|l|l|}
      \hline
      p  & n & m  & \multicolumn{4}{|c|}{\texttt{L-BFGS-B} results} & \multicolumn{4}{|c|}{\texttt{L-BFGS-B-NS} results} \\ \hline
      &  &  & iters. & \#fg & f & NPG & iters. & \#fg & f & NSVCHPG \\ \hline
      0.99 & 100 & 5 & 1 & 21 & 140004.324489439 & 7.79E+02 & 10000 & 29999 & 4706.5690751224 & 9.46E-01\\
      0.99 & 100 & 10 & 1 & 21 & 140004.324489439 & 7.79E+02 & 10000 & 29983 & 4706.5690446185 & 9.46E-01\\
      0.99 & 100 & 20 & 1 & 21 & 140004.324489439 & 7.79E+02 & 10000 & 29989 & 4706.5690446185 & 9.46E-01\\
      0.99 & 200 & 5 & 1 & 21 & 277216.896653442 & 1.10E+03 & 10000 & 29985 & 9332.020553172 & 9.46E-01\\
      0.99 & 200 & 10 & 1 & 21 & 277216.896653442 & 1.10E+03 & 10000 & 20009 & 9332.0205286749 & 6.56E+01\\
      0.99 & 200 & 20 & 1 & 21 & 277216.896653442 & 1.10E+03 & 10000 & 20007 & 9332.0182250141 & 8.21E+01\\
      0.99 & 1000 & 5 & 1 & 21 & 1374917.47396547 & 2.46E+03 & 10000 & 29993 & 46335.6319951054 & 9.46E-01\\
      0.99 & 1000 & 10 & 1 & 21 & 1374917.47396547 & 2.46E+03 & 10000 & 29993 & 46335.6319927555 & 9.46E-01\\
      0.99 & 1000 & 20 & 1 & 21 & 1374917.47396547 & 2.46E+03 & 10000 & 20009 & 46335.6319815415 & 1.92E+02\\
      0.99 & 5000 & 5 & 1 & 21 & 6863420.36052535 & 5.51E+03 & 10000 & 29995 & 231353.689126569 & 9.46E-01\\
      0.99 & 5000 & 10 & 1 & 21 & 6863420.36052535 & 5.51E+03 & 10000 & 29995 & 231353.689125989 & 9.46E-01\\
      0.99 & 5000 & 20 & 1 & 21 & 6863420.36052535 & 5.51E+03 & 10000 & 29995 & 231353.689125497 & 9.46E-01\\
      0.99 & 10000 & 5 & 1 & 21 & 13724048.9687287 & 7.79E+03 & 10000 & 20013 & 462626.260534309 & 4.91E+02\\
      0.99 & 10000 & 10 & 1 & 21 & 13724048.9687287 & 7.79E+03 & 10000 & 20013 & 462626.260533983 & 4.91E+02\\
      0.99 & 10000 & 20 & 1 & 21 & 13724048.9687287 & 7.79E+03 & 10000 & 20013 & 462626.260533741 & 4.91E+02\\
      \hline
    \end{tabular}
    \caption[A value where \texttt{L-BFGS-B-NS} is supposed to fail. $p = 0.99$]{Results for $p = 0.99$; similar to table \ref{p0999}}
    \label{p099}
  \end{center}
\end{table}

\begin{table}
  \tiny
  \begin{center}
    \begin{tabular}{|l|l|l|l|l|l|l|l|l|l|l|}
      \hline
      p  & n & m  & \multicolumn{4}{|c|}{\texttt{L-BFGS-B} results} & \multicolumn{4}{|c|}{\texttt{L-BFGS-B-NS} results} \\ \hline
      &  &  & iters. & \#fg & f & NPG & iters. & \#fg & f & NSVCHPG \\ \hline
      0.9 & 100 & 5 & 1 & 21 & 70247.1102599127 & 7.79E+02 & 10000 & 29985 & 3145.9378051899 & 7.82E+01\\
      0.9 & 100 & 10 & 1 & 21 & 70247.1102599127 & 7.79E+02 & 10000 & 20005 & 3145.9378011472 & 4.17E+02\\
      0.9 & 100 & 20 & 1 & 21 & 70247.1102599127 & 7.79E+02 & 10000 & 20007 & 3145.9375231332 & 2.66E+02\\
      0.9 & 200 & 5 & 1 & 21 & 137705.665344048 & 1.10E+03 & 10000 & 29983 & 6210.7940850593 & 5.70E-01\\
      0.9 & 200 & 10 & 1 & 21 & 137705.665344048 & 1.10E+03 & 10000 & 29987 & 6210.7940839115 & 5.70E-01\\
      0.9 & 200 & 20 & 1 & 21 & 137705.665344048 & 1.10E+03 & 10000 & 20007 & 6210.793392882 & 3.72E+02\\
      0.9 & 1000 & 5 & 1 & 21 & 677374.106017129 & 2.46E+03 & 10000 & 29997 & 30729.6443168733 & 2.49E+02\\
      0.9 & 1000 & 10 & 1 & 21 & 677374.106017129 & 2.46E+03 & 10000 & 29999 & 30729.6443166765 & 5.70E-01\\
      0.9 & 1000 & 20 & 1 & 21 & 677374.106017129 & 2.46E+03 & 10000 & 20013 & 30729.6443164162 & 1.58E+03\\
      0.9 & 5000 & 5 & 1 & 21 & 3375716.30938256 & 5.51E+03 & 10000 & 29993 & 153323.895471387 & 5.70E-01\\
      0.9 & 5000 & 10 & 1 & 21 & 3375716.30938256 & 5.51E+03 & 10000 & 29993 & 153323.895471342 & 5.70E-01\\
      0.9 & 5000 & 20 & 1 & 21 & 3375716.30938256 & 5.51E+03 & 10000 & 29993 & 153323.895471246 & 5.70E-01\\
      0.9 & 10000 & 5 & 1 & 21 & 6748644.0635896 & 7.79E+03 & 10000 & 29999 & 306566.709414405 & 5.70E-01\\
      0.9 & 10000 & 10 & 1 & 21 & 6748644.0635896 & 7.79E+03 & 10000 & 29999 & 306566.709414387 & 5.70E-01\\
      0.9 & 10000 & 20 & 1 & 21 & 6748644.0635896 & 7.79E+03 & 10000 & 29999 & 306566.709414375 & 5.70E-01\\
      \hline
    \end{tabular}
    \caption[A value where \texttt{L-BFGS-B-NS} is supposed to fail. $p = 0.9$]{Results for $p = 0.9$; similar to table \ref{p0999}}
    \label{p09}
  \end{center}
\end{table}

\chapter{Conclusions}

We conclude that the new code \texttt{L-BFGS-B-NS} works well on the modified Rosenbrock function, giving much better results than the original \texttt{L-BFGS-B} code for $p < 2$, when the function is not twice continuously differentiable, particularly for values of $p$ close to or equal to $1$. Even for $p < 1$, the computed function values seem reasonable, although we cannot verify this by \emph{NSVCHPG} condition since the gradients blow up when $f$ is not Lipschitz. We hope the new code will be useful to users who wish to solve large-scale non-smooth bound-constrained optimization problems.

\pagebreak
\pagebreak
\clearpage
